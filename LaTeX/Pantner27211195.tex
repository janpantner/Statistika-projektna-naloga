\documentclass[12pt, a4paper]{article}

\usepackage{statprojekt}
\newcommand{\naslov}{Projektna naloga iz Statistike}

\begin{document}

\renewcommand{\headheight}{20pt}

\maketitle
\thispagestyle{empty}

\newpage

\section{Kibergrad}

Preučujemo dohodke družin v mestu Kibergrad. Imamo informacije o 43.886 družinah, 
ki živijo v eni od štirih četrti: v severni četrti stanuje 10149 družin, v
vzhodni 10.390, v južni 13.457 in v zahodni 9.890. 

\newpage

\section{Lomljivost najlonskih palic}
Na vzorcu $280$ najlonskih palic preizkušamo njihovo lomljivost. Rezultati 
preizkusa so prikazani v sledeči tabeli.
\begin{table}[H]
    \centering
    \begin{tabular}{|l||l|l|l|l|l|l|}
        \hline
        \textbf{št.~lomov} & 0   & 1  & 2  & 3  & 4 & 5 \\ \hline
        \textbf{št.~palic} & 157 & 69 & 35 & 17 & 1 & 1 \\ \hline
    \end{tabular}
\end{table}
Privzemimo, da je število mest, na katerih se je palica 
zlomila porazdeljeno binomsko $\Bin(5, p)$ za določen neznan $p$.
Privzemimo tudi, da so palice med seboj neodvisne.

Pri teh predpostavkah je logaritem verjetja podan z
\[
    l(p, x) = \log\prod_{i=1}^n \binom{280}{x_i}p^{x_i}(1-p)^{5-x_i}.
\]
S pomočjo programa $\texttt{najlonske\_palice.py}$ numerično izračunamo, da je maksimum 
dosežen pri približno $p = 0{,}1421438163551369$. To nam predstavlja oceno $p$
po metodi največjega verjeta.


\end{document}