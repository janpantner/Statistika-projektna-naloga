\documentclass[12pt, a4paper]{article}

\usepackage{statprojekt}
\newcommand{\naslov}{Projektna naloga iz Statistike}

\addbibresource{stat.bib}

\begin{document}

\author{Jan Pantner \\
\small Profesor: doc.~dr.~Martin~Raič}
\date{September 2024}
\maketitle
\thispagestyle{empty}

\newpage

\tableofcontents
\newpage

\section{Kibergrad}

Preučujemo dohodke družin v mestu Kibergrad. Imamo informacije o $43.886$ družinah, 
ki živijo v eni od štirih četrti: v severni četrti stanuje $10.149$ družin, v
vzhodni $10.390$, v južni $13.457$ in v zahodni $9.890$. Pomagamo si 
s programom \texttt{kibergrad.py}.

Iz vsake četrti vzamemo enostavni slučajni vzorec velikosti $100$. 
Dohodke primerjamo s pomočjo škatel z brki, o katerih je več 
napisano v \cite[poglavje~10.6]{rice2007mathematical}.
Vzporedne škatle z brki so prikazane na sliki \ref{img:cetrti}.
\begin{figure}[H]
    \centering
    \includegraphics[width=12cm]{Slike/vse_cetrti.png}
    \caption{Škatle z brki za dohodke posamezne četrti.}
    \label{img:cetrti}
\end{figure}
Najprej opazimo, da je v severni četrti osamelec, ki močno izstopa. Prav tako so maksimum, tretji 
kvartil in mediana v severni četrti največji, vendar razlika ni dovolj velika, 
da bi lahko pri tako majhnem vzorcu sklepali, da so dohodki severne četrti 
najvišji. Zdi pa se, da so dohodki severne četrti malenkost višji kot dohodki 
južne četrti, pri kateri so vse prej omenjene vrednosti najmanjše. 

Zdi se, da v splošnem četrt ne vpliva močno na velikost dohodka.

Sedaj vzamemo še štiri enostavne slučajne vzorce velikosti $100$ 
iz severne četrti in pogledamo vzporedne škatle z brki za vzorce
iz severne četrti. Prikazane so na sliki \ref{img:sever}.
\begin{figure}[H]
    \centering
    \includegraphics[width=12cm]{Slike/sever.png}
    \caption{Škatle z brki dohodkov severne četrti.}
    \label{img:sever}
\end{figure}
Vse mediane so večje od $33.000$, njihovo povprečje je približno 
$36.500$. Zdi se, da lahko s precejšnjo sigurnostjo sklepamo, da je večina 
dohodkov višjih od $33.000$.

V novih vzorcih ne opazimo tako ekstremnih osamelcev kot pri prvem vzorcu. Večina 
osamelcev ni zelo oddaljenih od maksimuma.

Za konec si pogledamo še varianco dohodka, pojasnjeno s četrtmi in preostalo
(rezidualno) varianco. Naj bo $N$ velikost populacije, $N_i$ velikost 
populacije $i$-te četrti in $w_i$ velikostni delež $i$-te četrti. Naj bo
$\mu$ povprečni dohodek in $\sigma^2$ varianca dohodka Kibergrada,
$\mu_i$ in $\sigma_i^2$ pa zaporedoma povprečni dohodek in varianca 
dohodka $i$-te četrti. 
Pojma pojasnjene in nepojasnjene variance sta razložena v 
\cite{raic}. Zaporedoma sta določena s formulama
\[
    \sigma^2_B := \sum_{i=1}^{4} w_i(\mu_i - \mu)^2
    \quad \text{in} \quad
    \sigma^2_W := \sum_{i=1}^{4} w_i\sigma_i^2.
\]
Izračunamo, da je 
\[
    \sigma^2_B = 9.252.923
    \quad \text{in} \quad
    \sigma^2_W = 1.017.226.451.
\]
Pojasnjena varianca je velikokrat manjša od nepojasnjene.
To se ujema z opažanjem, da četrt ne vpliva močno na velikost dohodka.
\newpage

\section{Lomljivost najlonskih palic}
Na vzorcu $280$ najlonskih palic preizkušamo njihovo lomljivost. Rezultati 
preizkusa so prikazani v tabeli \ref{table:lomljivost}. Pri analizi 
si pomagamo s programom $\texttt{najlonske\_palice.py}$.
\begin{table}[H]
    \centering
    \begin{tabular}{|l||l|l|l|l|l|l|}
        \hline
        št.~lomov & 0   & 1  & 2  & 3  & 4 & 5 \\ \hline
        št.~palic & 157 & 69 & 35 & 17 & 1 & 1 \\ \hline
    \end{tabular}
    \caption{Rezultati preizkusa lomljivosti.}
    \label{table:lomljivost}
\end{table}
Privzamemo, da je število mest, na katerih se je palica 
zlomila, porazdeljeno binomsko $\Bin(5, p)$ za neznan parameter $p$.
Privzamemo tudi, da so palice med seboj neodvisne.

%-----------------------  Naloga (a) -----------------------%

Pri teh predpostavkah je logaritem verjetja podan z
\[
    l(p, x) = \log\prod_{i=1}^{280} \binom{5}{x_i}p^{x_i}(1-p)^{5-x_i}.
\]
Izračunamo, da je maksimum 
dosežen pri $\hat p = 0{,}14214$. Ta vrednost
nam predstavlja oceno $p$ po metodi največjega verjetja.

%-----------------------  Naloga (b) -----------------------%

Sedaj združimo zadnje tri vrednosti in preizkusimo ničelno domnevo, da 
je število mest, na katerih se je palica zlomila, porazdeljeno 
binomsko $\Bin(5, p)$, proti alternativni domneni, da ima število 
lomov katero drugo porazdelitev.
To storimo s Pearsonovim preizkusom hi kvadrat, ki je podrobneje
opisan v \cite[poglavje~9.5]{rice2007mathematical}.

Pearsonova testna statistika je
\[
    X^2 = \sum_{i=0}^{2}\frac{(x_i - nf_i(\hat p))^2}{nf_i(\hat p)},
\]
kjer je $x_0 = 157$, $x_1 = 69$, $x_2 = 35$, $x_3 = 19$ in
\[
    f_i(\hat p) = \begin{cases}
        \binom{5}{x_i}\hat p^{x_j}(1-\hat p)^{5-x_i}, &i \in \set{0,1,2} \\
        \sum_{j=3}^{5} \binom{5}{x_j}\hat p^{x_j}(1-\hat p)^{5-x_j}, &i = 3 
    \end{cases}.
\]
Izračunamo, da je $X^2 > 44$. Iz tabele kvantilov porazdelitve 
sledi, da ničelno domnevo zavrnemo tako 
pri stopnji tveganja $\alpha = 0{,}01$ kot tudi pri $\alpha = 0{,}05$. 

%-----------------------  Naloga (c) -----------------------%

Recimo sedaj, da imamo za $i = 1, \dots, 280$ neodvisna opažanja
$X_i \sim \Bin(m_i, p_i)$, kjer so parametri $m_i$ znani, 
$p_i$ pa neznani. Smiselno predpostavimo, da so vsi 
parametri $m_i$ vsaj $5$. S pomočjo razmerja verjetij preizkusimo 
ničelno domnevo, da so vsi parametri $p_i$ enaki, proti 
alternativni domnevi, da temu ni tako.

Verjetji sta enaka
\begin{align*}
    L_1 = \prod_{i=1}^{280} \binom{m_i}{x_i}p_i^{x_i}(1-p_i)^{m_i-x_i}, \\
    L_0 = \prod_{i=1}^{280} \binom{m_i}{x_i}p_0^{x_i}(1-p_0)^{m_i-x_i}.  
\end{align*}
Če logaritem verjetja $L_1$ odvajamo po $p_i$, dobimo
\[
    \frac{x_i}{p_i} + \frac{x_i-m_i}{1-p_i}.
\]
Odtod sledi, da je $\hat p_i = \frac{x_i}{m_i}$. Če je $x_i \in 
\set{0, m_i}$, tedaj vzamemo, da je tisti faktor verjetja enak $1$.
V primeru $L_0$ pa lahko numerično izračunamo maksimum verjetja in 
dobimo $\hat p_0$.

Na danih podatkih dobimo
\[
    2\log\Lambda = 2\log\frac{\prod_{i=158}^{279} \binom{m_i}{x_i}p_i^{x_i}(1-p_i)^{m_i-x_i}}
    {\prod_{i=1}^{280} \binom{m_i}{x_i}p_0^{x_i}(1-p_0)^{m_i-x_i}}
    = 444{,}489.
\]
Wilksovega izreka tukaj ne moremo uporabiti, saj zahteva, da 
so spremenljivke enako porazdeljene.
Namesto tega uporabimo metodo \textit{bootstrap}.
Med $10.000$ simuliranimi vrednostmi preizkusne statistike pri 
ničelni domnevi jih $0$ preseže vrednost $444{,}489$, torej 
lahko ničelno domnevo ponovno zavrnemo.

\newpage

\section{Spreminjanje temperature v Ljubljani}

Preučujemo spreminjanje temperature v Ljubljani s 
pomočjo podatkov, izmerjenih mesečno v letih od $1994$ do $2023$.
Pomagamo si s programom \texttt{temperature.py}.

Najprej predpostavimo linearni trend in sinusno nihanje s periodo 
enega leto. To nam da model
\[
    y_{l,m} = Ax_l + B\sin(x_m\pi/6 + \delta) + C
\]
oziroma
\[
    y_{l,m} = Ax_l + B\sin(x_m\pi/6) + C\cos(x_m\pi/6) + D,
\]
kjer je $x_l$ leto meritve, $x_m$ mesec meritve, $y_{l,m}$ pa 
izmerjena temperatura v tem mesecu tega leta. Več o sinusnem 
modelu v \cite{wiki:Sinusoidal_model}. Alternativno je model 
določen z
\[
    Y = X_A\beta_A,
\]
kjer je $\beta_A^\top = \begin{bmatrix}
    D & C & B & A
\end{bmatrix}$ in
\[
    X_A = \begin{bmatrix}
        1 & 1 & 0 & 1 \\
        1 & 2 & \sin\left(\pi/6\right) & \cos\left(\pi/6\right) \\
        \vdots & \vdots & \vdots & \vdots \\
        1 & 12 &  \sin\left(11\pi/6\right) & \cos\left(11\pi/6\right) \\
        1 & 13 & 0 & 1 \\
        \vdots & \vdots & \vdots & \vdots \\
        1 & 24 &  \sin\left(11\pi/6\right) & \cos\left(11\pi/6\right) \\
        1 & 25 &  0 & 1 \\
        \vdots & \vdots & \vdots & \vdots \\
        1 & 360 & \sin\left(11\pi/6\right) & \cos\left(11\pi/6\right)\\
        \end{bmatrix}.
\]
Torej je $X_A$ matrika velikosti $360 \times 4$. Predpostavimo, da je 
naš model Gaussov. Tedaj je ocena za $\beta_A$ po metodi največjega 
verjetja enaka
\[
    \hat \beta_A = (X_A^\top X_A)^{-1} X_A^\top Y.
\]
Izračunamo
\[
    \beta_A^\top = \begin{bmatrix}
        10.784 & 0.047 & 0.056 & -10.375
    \end{bmatrix}.
\]
Prva komponenta nam pove, da imamo pozitiven trend, kar se sklada 
s teorijo globalnega segrevanja. Kako dober je model, lahko ocenimo s pomočjo slike 
\ref{png:prvi}.
\begin{figure}[H]
    \centering
    \includegraphics[width=14cm]{Slike/prvi_model.png}
    \caption{Siva krivulja označuje izmerjene podatke, zelena pa oceno 
    prvega modela.}
    \label{png:prvi}
\end{figure}
Alternativno predpostavimo linearni trend in spreminjanje temperature za vsak mesec
v letu posebej, torej
\[
    X_B = \begin{bmatrix}
        1 & 1 & 0 & 0 & \cdots & 0 & 0 \\
        2 & 0 & 1 & 0 & \cdots & 0 & 0 \\
        \vdots & & & & \ddots & & & \\
        12 & 0 & 0 & 0 & \cdots & 0 & 1 \\
        13 & 1 & 0 & 0 & \cdots & 0 & 0 \\
        \vdots & & & & \ddots & & & \\
        24 & 0 & 0 & 0 & \cdots & 0 & 1 \\
        25 & 1 & 0 & 0 & \cdots & 0 & 0 \\
        \vdots & & & & \ddots & & & \\
        360 & 0 & 0 & 0 & \cdots & 0 & 1 
    \end{bmatrix}.
\]
To je matrika velikosti $360 \times 13$. V tem primeru je
\footnotesize
\setcounter{MaxMatrixCols}{20}
\[
    \beta_B^\top = \begin{bmatrix}
        0.005 & 0.32 & 2.19 & 6.35 & 10.72 & 15.32 & 
        19.59 & 21.24 & 20.50 & 15.46 & 10.89 & 5.83 & 0.93
    \end{bmatrix}.
\]
\normalsize
Prva komponenta nam pove, da imamo pozitiven linearen trend, 
iz ostalih komponent, pa je lepo razvidno kako se temperature
spreminjajo skozi mesece. Vidimo na primer, da model 
predvidi najnižje temperature pozimi in najvišje poleti.
Model je prikazan na sliki \ref{png:drugi}.
\begin{figure}[H]
    \centering
    \includegraphics[width=14cm]{Slike/drugi_model.png}
    \caption{Siva krivulja označuje izmerjene podatke, modra pa oceno 
    tretjega modela.}
    \label{png:drugi}
\end{figure}
Drugi model je širši od prvega. Preizkusimo model $A$ znotraj 
modela $B$, torej preizkusimo ničelno domnevo, da velja model $A$, 
proti alternativni, da velja model $B$ z odvzetim podmodelom $A$.
To lahko storimo na podlagi statistike
\[
    F := \frac{(\RSS_A - \RSS_B)(n-p)}{\RSS_B (p-q)},
\]
kjer sta $p$ in $q$ prostorski stopnji modelov $B$ in $A$. 
Ničelno domnevo zavrnemo, če je 
\[
    F \ge F^{-1}_{\fish(p-q,n-p)}(1-\alpha),
\]
kjer je $\alpha$ stopnja tveganja. 
Izpeljava je na voljo v \cite{raic2}. 

V našem primeru je $p = 13$ in $q = 4$. Izračunamo 
\[
    F = 1{,}621, \quad
    F^{-1}_{\fish(9,347)}(1-0{,}05) = 1{,}907 \quad \text{in} \quad
    F^{-1}_{\fish(9,347)}(1-0{,}01) = 2{,}459.
\]
Ničelne domneve torej ne moremo zavrniti.

\subsection*{Določanje optimalnega modela z Akaikejevo informacijo}

Pri izbiri boljšega modela si lahko pomagamo z Akaikejevo informacijo, 
ki je v primeru linearne regresije in Gaussovega modela enaka
\[
    \AIC := 2p + n \ln \RSS,
\]
kjer je $p$ število parametrov, $n$ število opažanj in $\RSS = \norm{Y - X\hat \beta}$.

Akaikejeva informacija prvega modela je enaka $1262$,
drugega pa $1265$. Torej ima model $A$ malenkost
manjšo Akaikejevo informacijo in je primernejši.

\newpage

\nocite{*}
\printbibliography
\addcontentsline{toc}{section}{Literatura}

Kvantili Fisherjeve porazdelitve so bili izračunani s kalkulatorjem Stat Trek,
dostopnim na \url{https://stattrek.com/online-calculator/f-distribution}.


\end{document}